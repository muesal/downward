%! Author = salom
%! Date = 11.09.2020

\chapter{Strengthening Potential Heuristics}\label{ch:strengthening-potential-heuristics}

\citeauthor{fivser2020strengthening} proposed to improve potential heuristics with mutexes and disambiguations
This chapter contains the changes of the transformation of the planning task into TNF and the adaption of the optimization functions.
It shows how the equations which were implemented were derived.

\section{Potential Heuristics}\label{sec:potential-heuristics}

When~\citeauthor{pommerening2015non} first introduced potential heuristics, they showed that two inequalities are sufficient to proof admissibility.

\begin{theorem}
    \label{theorem:theorem 5} % Theorem 6
    Let $\Pi = \langle \mathcal{V}, \mathcal{O}, I, G \rangle$ denote a planning task, $\mathtt{P}$ a
    potential function, and for every operator $o\in\mathcal{O}$, let
    $\mathrm{pre}^*(o)=\{\langle V, \mathrm{pre}(o)[V]\rangle |V\in \mathrm{vars(pre}(o))\cap\mathrm{vars(eff}(o))\}$ and
    $\mathrm{vars}^*(o)=\mathrm{vars(eff}(o))\setminus\mathrm{vars(pre}(o))$. If
    \[\sum_{f\in G}\mathtt{P}(f)+\sum_{V\in\mathcal{V}\setminus\mathrm{vars}(G)}\max_{f\in\mathcal{F}_V}\mathtt{P}(f)\leq0\label{eq:1}\tag{1}\]
    and for every operator $o\in\mathcal{O}$ it holds that
    \[\sum_{f\in\mathrm{pre}^*(o)}\mathtt{P}(f)+\sum_{V\in\mathrm{vars}^*(o)}\max_{f\in\mathcal{F}_V}\mathtt{P}(f)-\sum_{f\in\mathrm{eff}(o)}\mathtt{P}(f)\leq c(o)\label{eq:2}\tag{2}\]
    then the potential heuristic for $\mathtt{P}$ is admissible.
\end{theorem}


Eq.~\eqref{eq:1} of the theorem by~\citeauthor{fivser2020strengthening} assures goal-awareness of the potential heuristic.
As all variables are assigned in the goal state, the potential of one fact per variable has to be summed up.
For the variables $v\in\text{vars}(G)$ we can simply use the potentials of their respective facts.
Meanwhile we assume the worst case for the other variables, by using the maximal potential over their facts, as we do not know what fact they are assigned.

Eq.\eqref{eq:2} assures consistency.
Recall the general consistency equation $h(s)\leq h(o\llbracket s\rrbracket)+\text{c}(o)$.
It can be rewritten as $h(s)-h(o\llbracket s\rrbracket)\leq\text{c}(o)$.
As the facts which do not occur in the effect are the same in both $s$ and $o\llbracket s\rrbracket$ we cam leave them aside. % da substraktion
For $s$ we know what facts of the variables of the preconditions are assigned and sum the potentials of those which are also in effect.
For the variables which are in effect but not in the precondition we proceed similar to ~\eqref{eq:1}, as we do not know their values.
The potentials of the facts in the effect can be used without modification for  $o\llbracket s\rrbracket$.

The Advantage of this equations is that they are not state-dependent, even though they do not tell us what exactly the potentials should be.
However, they can be used as the constraints for a linear program (\textbf{LP}), the solution of which is a potential function that forms an admissible potential heuristic.
More about this in~\ref{subsec:transition-normal-form}.

\subsection{Generalize with Mutexes}\label{subsec:generalize-with-mutexes}
\mytodo { some general writng about how mutexes can be used for narrowing down domains}

Using this [above],~\ref{theorem:theorem 5} can be generalized by the following theorem.

\begin{theorem}
    \label{theorem:7}
    Let $\Pi = \langle \mathcal{V}, \mathcal{O}, I, G \rangle$ denote a planning task with facts $\mathcal{F}$, and let $\mathtt{P}$ denote a potential function, and
    \begin{enumerate}[(i)]
        \item for every variable $V\in\mathcal{V}$, let $G_V\subseteq\mathcal{F}_V$ denote a disambiguation of $V$ for $G$ s.t. $|G_V|\geq1$, and
        \item for every operator $o\in\mathcal{O}$ and every variable $V\in\mathrm{vars(eff}(o))$, let $E^o_V\subseteq\mathcal{F}_V $ denote a disambiguation of $V$ for $\mathrm{pre}(o)$ s.t. $|E^o_V|\geq1$.
    \end{enumerate}

    If
    \[\sum_{V\in\mathcal{V}}\max_{f\in G_V}\mathtt{P}(f)\leq0\label{eq:3}\]
    and for every operator $o\in\mathcal{O}$ it holds that
    \[\sum_{V\in\mathrm{vars(eff}(o))}\max_{f\in E^o_V}\mathtt{P}(f) - \sum_{f\in\mathrm{eff}(o)}\mathtt{P}(f)\leq\mathrm{c}(o)\label{eq:4}\]
    then the potential heuristic $\mathtt{P}$ is admissible.
\end{theorem}

\citeauthor{fivser2020strengthening} proof the theorem by showing that equations~\eqref{eq:3} and~\eqref{eq:4} are generalizations of equations~\eqref{eq:1} and~\eqref{eq:2} respectively.

$G_V$ is the disambiguation of $V\in\mathcal{V}$ for $G$, i.e., it holds all facts of $\mathcal{F}_V$ which are not a mutex with any of tha facts $f\in G$.
If it is empty for any of the variables, then the problem is unsolvable.
$E^o_V$ is the disambiguation of $V\in\text{vars(eff}(o))$ for $\text{pre}(o)$, $o\in\mathcal{O}$.
In words, for all variables which are affected by the application of $o$, remove facts which are a mutex with any of the facts in the precondition.
$o$ is not applicable in any (partial) state, if $E^o_V$ is empty for any  $V\in\text{vars(eff}(o))$.

\mytodo{anything else for this subsection? something kind of concluding is missing}


\subsection{Transition Normal Form}\label{subsec:transition-normal-form}
\mytodo{What is TNF?}

\mytodo{does some of this belong to the background?}


\begin{comment}
    %TODO: this is for the LP solver, from one of helmerts papers?
    \begin{definition}
        Let $f$ be a solution to the following LP:
        Maximize $\mathrm{opt}$ subject to $\sum_{V\in\mathcal{V}}\mathtt{P}_{\langle V, s[V]\rangle}\leq0$ and
        $\sum_{V\in\mathrm{vars(eff}(o))}(\mathtt{P}_{\langle V, \mathrm{pre}(o)[V]\rangle}-\mathtt{P}_{\langle V, \mathrm{eff}(o)[V]\rangle})\leq\mathrm{c}(o)$
        for all $o\in\mathcal{O}$, where the objective function $\athrm{opt}$ can be chosen arbitrarily.
        Then the function $\mathrm{pot}_{\mathrm{opt}}(\langle V,v\rangle)=f()$
    \end{definition}
\end{comment}


\section{Optimization}\label{sec:optimization}


