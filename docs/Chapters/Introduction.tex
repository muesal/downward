%! Author = salom
%! Date = 18.09.2020
\chapter{Introduction}\label{ch:introduction}

One dimensional potential heuristics assign a potential, i.e., a numerical value, to each fact of a classical planning problem.
They can be used in a heuristic search, the most common approach to solve such problems.
The potentials are obtained with a Linear Program, which optimizes the potentials according to an optimization function.
Different optimization functions yield different heuristics.
The heuristic value of a state is the sum over the potentials of the fact in this state.

In this thesis, we reproduce the work of~\cite{fivser2020strengthening} who proposed to strengthen potential heuristics with mutexes and disambiguations.
Mutexes are sets of facts, which can never appear together in any reachable state.
They can be used to build disambiguation sets for partial states.
They contain all remaining facts which can be assigned to this partial state, i.e., which are not mutex with any of the facts in the partial state.

As introduced by~\cite{fivser2020strengthening} we use mutexes and disambiguations to build a less restricted Linear Program.
This does lead to better heuristics but the computation is too expensive and slightly less problems are solved.

Next, we use mutexes and disambiguations to strengthen the optimization functions.
This leads to good single and ensemble heuristics, however not more problems can be solved, than with the non-strengthened optimization functions.

Further, we add the additional constraint on the initial state to the constraints of the LP, as~\cite{fivser2020strengthening} proposed.
This does indeed enhance the performance, as the resulting heuristics solve more problems as the same heuristics computed without the additional constraint.

Taking this idea one step further, we add additional constraints on random states.
Therefore, random states are generated with random walks, and constraints gained by optimizing the LP for this states are added.
These additional constraints do not enhance the performance of the resulting heuristics in comparison to the constraint on the initial state.
They are, however, better than using no additional constraint at all, and could be further enhanced.

Last, we compare our results to the evaluation of~\cite{fivser2020strengthening}.
It turns out, that the translation and pre processing of the planning tasks they use, as well as their way for generating of the potentials, are faster than ours.